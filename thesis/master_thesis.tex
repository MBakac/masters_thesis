\documentclass[masterthesis]{fer}
% Add the option upload to generate the final version which is uploaded to FERWeb
% Dodaj opciju upload za generiranje konačne verzije koja se učitava na FERWeb
% \documentclass[masterthesis,upload]{fer} % TODO: upload

% D.K. imal 44 stranice sadržaja
% I.K. imal 31 stranicu sadržaja
% M.J.K. imal 48 stranicu sadržaja s puno slika
% M.H. imal 45 stranice sadržaja

\usepackage{blindtext}

\usepackage[croatian]{babel}

%--- THESIS INFORMATION / PODACI O RADU ----------------------------------------

% Title in English / Naslov na engleskom jeziku
\title{A Transformer model for chord segmentation and recognition} % TODO: check

% Title in Croatian / Naslov na hrvatskom jeziku
\naslov{Model za segmentaciju i prepoznavanje akorda s arhitekturom transformera}

% Thesis number / Broj rada
\brojrada{1186}

% Author / Autor
\author{Martin Bakač}

% Mentor 
\mentor{Prof.\@ Ivan Đurek}

% Date in English / Datum rada na engleskom jeziku
\date{February, 2026}

% Date in Croatian / Datum rada na hrvatskom jeziku
\datum{veljača, 2026.}

%-------------------------------------------------------------------------------


\begin{document}


% Titlepage is automatically generated / Naslovnica se automatski generira
\maketitle


%--- THESIS ASSIGNMENT / ZADATAK -----------------------------------------------

% Thesis assignment is included from external file / Zadatak se ubacuje iz vanjske datoteke
% Enter the filename of the PDF downloaded from FERWeb / Upiši ime PDF datoteke preuzete s FERWeb-a
\zadatak{../zadatak.pdf}


%--- ACKNOWLEDGMENT / ZAHVALE --------------------------------------------------

\begin{zahvale}
  % Write in the acknowledgment / Ovdje upišite zahvale
  Hvala obitelji i prijateljima na podršci tokom studija.
  \\
  Zahvaljujem društvu Digital Media d.o.o. na usutupljenom hardveru.
\end{zahvale}


% Page numbering starts from here / Odovud započinje numeriranje stranica
\mainmatter


% Table of contents is automatically generated / Sadržaj se automatski generira
\tableofcontents


%--- INTRODUCTION / UVOD -------------------------------------------------------
\chapter{Uvod}
\label{chp:uvod}
\input{Chapters/introduction}

%--- THESIS RELATED CHAPTERS / POGLAVLJA VEZANA UZ ZADATAKA --------------------
\chapter{Segmentacija i prepoznavanje akorda}
\label{cha:segmentacija_i_prepoznavanje_akorda}
%--- Segmentacija i prepoznavanje akorda ----------------

\section{Glazbena pozadina problema}
\subsection{Akordi u glazbi}

\section{Rasčlamba problema}

\section{Skupovi podataka}
\subsection{Značajke korištene za učenje}

\section{Primjena i komercijalizacija modela}


\chapter{Modeli s arhitekturom transformera}
\label{cha:modeli_s_arhitekturom_transformera}
\input{Chapters/transformer_models}

\chapter{Korisničko sučelje}
\label{cha:korisnicko_sucelje}
\input{Chapters/user_interface}



%--- CONCLUSION / ZAKLJUČAK ----------------------------------------------------
\chapter{Zaključak}
\label{chp:zakljucak}


%--- REFERENCES / LITERATURA ---------------------------------------------------

% References are automatically generated from the supplied .bib file / Literatura se automatski generira iz zadane .bib datoteke
% Enter the name of the BibTeX file without .bib extension / Upiši ime BibTeX datoteke bez .bib nastavka
\bibliography{references}



%--- ABSTRACT / SAŽETAK --------------------------------------------------------

% Abstract in English
\begin{abstract}
  Enter the abstract in English.
\end{abstract}

\begin{keywords}
  the first keyword; the second keyword; the third keyword
\end{keywords}


% Sažetak na hrvatskom
\begin{sazetak}
  Unesite sažetak na hrvatskom.
\end{sazetak}

\begin{kljucnerijeci}
  prva ključna riječ; druga ključna riječ; treća ključna riječ
\end{kljucnerijeci}



%--- APPENDIX / PRIVITCI -------------------------------------------------------

% All following chapters will be denoted with an appendix and a letter / Sva poglavlja koja slijede će biti označena slovom i riječi privitak
\backmatter

\chapter{The Code}


\end{document}
